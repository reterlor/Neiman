% !TeX spellcheck = ru_RU
\documentclass[twoside,12pt]{article}
\usepackage{titlesec}
\usepackage{enumitem}
\usepackage[T2A,T1]{fontenc}
\usepackage[utf8]{inputenc}
\usepackage[russian]{babel}
\usepackage{amsmath,amssymb,amsthm}
\usepackage{fancyhdr}
\pagestyle{fancy}
\fancyhf{}
\fancyfoot{}
\renewcommand{\headrulewidth}{0pt}
\setlength{\headheight}{15pt}
\usepackage[a4paper,bindingoffset=0.2in,%
left=1in,right=1in,top=1in,bottom=1in,%
footskip=.10in]{geometry}

\newcommand{\eps}{\varepsilon}
\newcommand{\inprod}[1]{\left\langle #1 \right\rangle}


\newcommand{\handout}[5]{
	\noindent
	\begin{center}
		\framebox{
			\vbox{
				\hbox to 5.78in { {\bf Научно-исследовательская практика} \hfill #2 }
				\vspace{4mm}
				\hbox to 5.78in { {\Large \hfill #5  \hfill} }
				\vspace{2mm}
				\hbox to 5.78in { {\em #3 \hfill #4} }
			}
		}
	\end{center}
	\vspace*{4mm}
}

\newcommand{\lecture}[4]{\handout{#1}{#2}{#3}{#4}{Cистема верстки LaTeX  #1}}





\begin{document}
\lecture{}{Лето 2020}{}{Нейман Даниил}	
\newpage	
	
\renewcommand{\headrulewidth}{0pt}
\lhead{\textbf{Раздел 13-3}}
\chead{\textbf{Конечные Цепные Дроби}}
\rhead{\textbf{303}}
	\indent{Так как цепные дроби громоздки для печати или записи,
		мы станем обозначать цепную дробь записью которая показывает
		её частичные коэффициенты; допустим, записью $[ a_0 ; a_1 , ..., a_n]$.
		В этом обозначении расширение 19/51 записывается в виде }
	
	\begin{center}
		[0; 2, 1, 2, 6]
	\end{center}
	
	\noindent{и для 172/51 = 3 + 19/51 в виде}
	
	\begin{center}
		[3; 2, 1, 2, 6].
	\end{center}
	
	\noindent{Первичное целое число в записи $[ a_0 ; a_1 , ..., a_n]$ будет равно нулю  когда
		значение дроби положительно, но меньше единицы.}\\
	\indent{Представление рационального числа в виде конечной простой цепной дроби не единственно: как только представление было получено,
		мы всегда можем изменить последний элемент. Если $a_n > 1$, тогда }
	
	\begin{center}
		$a_n$ = ($a_n$ — 1) + 1 = ($a_n$ — 1) + $\frac{1}{1}$,
	\end{center}
	
	\noindent{где $a_n$ — 1, положительное целое число, следовательно}
	
	\begin{center}
		$[ a_0 ; a_1 , ..., a_n] = [ a_0 ; a_1 , ..., a_n - 1, 1]$.
	\end{center}
	
	\noindent{с другой стороны, если $a_n$ = 1, значит}
	
	\begin{center}
		$a_{n-1} + \frac{1}{a_n} = a_{n-1} + \frac{1}{1} = a_{n-1} + 1$,
	\end{center}
	
	\noindent{так, что}
	
	\begin{center}
		$[ a_0 ; a_1 , ..., a_{n-1}, a_n] = [ a_0 ; a_1 , ..., a_{n-2},a_{n-1}+1]$,
	\end{center}
	
	\noindent{Любое рациональное число имеет два представления в виде простой 
		цепной дроби, одно с четным количеством неполных частных,
		и другое, с нечетным количеством неполных частных
		(оказывается что это единственные два представления).
		В случае 19/51,}
	
	\begin{center}
		$19/51 = [0; 2, 1, 2, 6] = [0; 2, 1, 2, 5, 1]$.
	\end{center}
	
	\noindent{\textbf{Пример 13-1}}
	\begin{quote}
	\noindent{Вернемся к последовательности Фибоначчи и рассмотрим коэффициенты двух последовательных чисел Фибоначчи (рациональные числа $u_{n+1}$/$u_n$) записанные в виде простых цепных дробей. Из сказанного}
	\newpage
		
	\renewcommand{\headrulewidth}{0pt}
	\lhead{\textbf{304}}
	\chead{\textbf{Числа Фибоначчи и Цепные Дроби}}
	\rhead{\textbf{Глава 13}}
\noindent{ранее, Алгоритм Евклида для наибольшего общего делителя $u_n$ и $u_{n+1}$ производит $n-1$ уравнений}
	\begin{gather}
 \begin{align}
 &u_{n+1}=1\cdot u_n+u_{n-1},&&\notag\\
 &\quad u_n=1 \cdot u_{n-1}+u_{n-2},&&\notag\\
 &\quad\vdots&&\notag\\
 &\quad u_4=1 \cdot u_3+u_2, &&\notag\\
 &\quad u_3=2 \cdot u_2+0.\notag&&
 \end{align}
   \end{gather}
\noindent{Поскольку коэффициенты полученные алгоритмом становятся неполными 
частными цепной дроби, запишем $$u_{n+1}/u_n=[1;1,1,...,1,2].$$
Но $u_{n+1}/u_n$ так же представляется в виде цепной дроби имеющей на 
один неполный частный больше чем $[1;1,1,...,1,2]$; а именно,$$u_{n+1}/u_n=[1;1,1,...,1,1,1],$$ где число 1 встречается $n+1$ раз. Таким образом, дробь $u_{n+1}/u_n$
имеет расширение в виде цепной дроби которое легко описать:
цепная дробь с  $n$ неполныx частныx равных 1.}
\end{quote}

\indent{В заключении, мы хотели бы показать как теория о цепных дробях
может быть применена для решения линейных Деофантовых уравнений. 
Для этого потребуется знание нескольких фактов касающихся 
"подходящих дробей" \! цепной дроби, давайте начнем с их доказательства.}

\begin{quote}
\noindent{\textsc{Определение 13-2.} Цепная дробь полученная из отделения от $[a_0;a_1,...,a_n]$
неполных частных после $k$-ого элемента $a_k$ называется $k$-ой подходящей дробью и обозначается $C_k;$
\begin{align}
C_k=[a_0;a_1,...,a_k], \tag*{$(1\leqslant k\leqslant n).$}
\end{align}

Нулевой подходящей дробью $C_0$ будем обозначать $a_0$. }\\
\end{quote}

\indent{Стоит подметить, что для $k<n$, если $a_k$ заменяется значением $a_k+1/a_{k+1}$, тогда подходящая дробь $C_k$ становится подходящей дробью $C_k+1$;
$$[a_0;a_1,...,a_{k-1},a_k+1/a_{k+1}]=[a_0;a_1,...,a_{k-1},a_k,a_{k+1}]=C_{k+1}.$$}

	\newpage

\renewcommand{\headrulewidth}{0pt}
\lhead{\textbf{Раздел 13-3}}
\chead{\textbf{Конечные Цепные Дроби}}
\rhead{\textbf{305}}
\noindent{Вряд ли нуждается в замечании, что последняя подходящая дробь $C_n$ всегда равна рациональному числу первоначальной цепной дроби. }\\
\indent{Возвращаясь к нашему примеру $19/51=[0;2,1,2,6],$ последователые \\подходящие дроби равны 
\begin{gather}
\begin{align}
&C_0=0,&& \notag\\
&C_1=[0;2]=0+\frac{1}{2}=\frac{1}{2},&&\notag\\
&C_2=[0;2,1]=0+\frac{1}{2+\frac{1}{1}}=\frac{1}{3},&& \notag\\
&C_3=[0;,2,1,2]=0+\frac{1}{2+\frac{1}{1+\frac{1}{2}}}=\frac{3}{8},&&\notag\\
&C_4=[0;2,1,2,6]=19/51.&& \notag
\end{align}
\end{gather}
Кроме последней подходящей дроби $C_4$, они либо больше, либо меньше 19/51, при этом каждая последующая подходящая дробь ближе к $19/51$ чем предыдущая.}
\indent{Значительную часть труда в вычислении подходящих дробей цепной дроби $[a_0;a_1,...,a_n]$ можно избежать при помощи формул для их числителей и знаменателей. Для этого, определим числа $p_k$ и $q_k$ $(k=0,1,...,n)$ таким образом, что}

\begin{center}
$\begin{array}{ll}
p_0=a_0 & q_0=1 \\
p_1=a_1a_0+1 & q_1=a_1 \\
p_k=a_kp_{k-1}+p_{k-2} & q_k=a_kq_{k-1}+q_{k-2}
\end{array}$
\end{center}
\noindent{для $k=2,3,...,n.$}\\
\indent{Прямое вычисление показывает, что первые несколько подходящих дробей $[ a_0 ; a_1 , ..., a_n]$ равны:}
\begin{gather}
\begin{align}
&C_0=a_0=\frac{a_0}{1}=\frac{p_0}{q_0},&& \notag\\
&C_1=a_0+\frac{1}{a_1}=\frac{a_1a_0+1}{a_1}=\frac{p_1}{q_1},&&\notag\\
&C_2=a_0+\frac{1}{a_1+\frac{1}{a_2}}=\frac{a_2(a_1a_0+1)+a_0}{a_2a_1+1}=\frac{p_2}{q_2}.&&\notag
\end{align}
\end{gather}
\noindent{Успех зависит от того, сможем ли мы показать что это взаимоотношение поддерживается и далее. Это содержание Теоремы 13-6.}

\newpage
\renewcommand{\headrulewidth}{0pt}
\lhead{\textbf{306}}
\chead{\textbf{Числа Фибоначчи и Цепные Дроби}}
\rhead{\textbf{Глава 13}}
\noindent{\textsc{Теорема 13-6. \textit{\textbf{ k-ая подходящая дробь простой цепной дроби $[a_0;a_1,...,a_n]$ имеет значение}}}}
\begin{align}
C_k=p_k/q_k \tag*{$(0\leqslant k\leqslant n).$}
\end{align}
\noindent{\textit{\textbf{Доказательство: }} Рассмотренные выше замечания указывают, что теорема выполняется для $k=0,1,2.$
Допустим, что она выполняется и для $k=m$, где $2\leqslant m<n$; то есть для этого m,}
\makeatletter
\newcommand{\LeftEqNo}{\let\veqno\@@leqno}
\makeatother
\begin{equation}\LeftEqNo
C_m=p_m/q_m =\frac{a_mp_{m-1}+p_{m-2}}{a_mq_{m-1}+q_{m-2}}.\tag{\textasteriskcentered}
\end{equation}
\noindent{Обратим внимание, что числа $p_{m-1},q_{m-1},p_{m-2},q_{m-2}$ зависят от первых $m-1$ неполных частных $a_1,a_2,...,a_m-1,$ следовательно не зависят от $a_m$. Таким образом формула $(\textasteriskcentered)$ остается действительной если заменить $a_m$ значением $a_m+1/a_{m+1}:$
$$\bigg[a_0;a_1,...,a_{m-1},a_m+\frac{1}{a_{m+1}}\bigg]=\frac{\Big(a_m+\frac{1}{a_{m+1}}\Big)p_{m-1}+p_{m-2}}{\Big(a_m+\frac{1}{a_{m+1}}\Big)q_{m-1}+q_{m-2}}$$
Из вышесказанного, цель этой замены в том, чтобы изменить $C_m$ в подходящую дробь $C_{m+1}$ так, что
\begin{equation}
\begin{aligned}
C_{m+1}&=\frac{\Big(a_m+\frac{1}{a_{m+1}}\Big)p_{m-1}+p_{m-2}}{\Big(a_m+\frac{1}{a_{m+1}}\Big)q_{m-1}+q_{m-2}}\notag\\
&=\frac{a_{m+1}(a_mp_{m-1}+p_{m-2})+p_{m-1}}{a_{m+1}(a_mq_{m-1}+q_{m-2})+q_{m-1}}\notag\\
&=\frac{a_{m+1}p_m+p_{m-1}}{a_{m+1}q_m+q_{m-1}.}
\end{aligned}
\end{equation}\\
Но это именно тот вид, который теорема должна принимать при $k=m+1$. Значит по методу математической индукции теорема доказана.}\\
\indent{Рассмотрим это на конкретном примере. $19/51 = [0;2,1,2,6]:$}

\begin{center}
	$\begin{array}{lcl}
	p_0=0 &$ и $& q_0=1, \\
	p_1=0\cdot2+1=1 & & q_1=2, \\
	p_2=1\cdot1+0=1 & & q_2=1\cdot2+1=3, \\
	p_3=2\cdot1+1=3 & & q_3=2\cdot3+2=8, \\
	p_4=6\cdot3+1=19 & & q_4=6\cdot8+3=51. \\
	\end{array}$
\end{center}
\pagebreak
	\end{document}